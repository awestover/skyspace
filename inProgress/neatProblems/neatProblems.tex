\documentclass{article}[11pt]
\usepackage[subtle]{savetrees}
\usepackage[left=1in, right=1in, top=1in, bottom=1in]{geometry}

\usepackage{amsthm}
\usepackage{amssymb}
\usepackage{amsmath}
\usepackage{mathtools}

\usepackage{fancyhdr}
\pagestyle{fancy}
\lhead{Alek Westover}
\rhead{}

\DeclareMathOperator{\E}{\mathbb{E}}
\DeclareMathOperator{\Var}{\text{Var}}
\DeclareMathOperator{\img}{Im}
\DeclareMathOperator{\polylog}{\text{polylog}}
\DeclareMathOperator{\st}{\text{ such that }}
\newcommand{\norm}[1]{\left\lVert#1\right\rVert}
\newcommand{\interior}[1]{%
  {\kern0pt#1}^{\mathrm{o}}%
}

\newcommand{\contr}[0]{\[ \Rightarrow\!\Leftarrow \]}
\newcommand{\defeq}{\vcentcolon=}
\newcommand{\eqdef}{=\vcentcolon}

\newtheorem{fact}{Fact}
\newtheorem{definition}{Definition}
\newtheorem{remark}{Remark}
\newtheorem{proposition}{Proposition}
\newtheorem{lemma}{Lemma}
\newtheorem{corollary}{Corollary}
\newtheorem{theorem}{Theorem}

\usepackage{tcolorbox}
\newenvironment{lem}[1][Lemma] { \begin{tcolorbox}[colback=green!10,coltitle=green!20!black,colframe=green!25,title=#1] } { \end{tcolorbox} }
\newenvironment{cor}[1][Corollary] { \begin{tcolorbox}[colback=green!10,coltitle=green!20!black,colframe=green!25,title=#1] } { \end{tcolorbox} }
\newenvironment{prop}[1][Proposition] { \begin{tcolorbox}[colback=green!10,coltitle=green!20!black,colframe=green!25,title=#1] } { \end{tcolorbox} }
\newenvironment{defn}[1][Definition] { \begin{tcolorbox}[colback=orange!10,coltitle=orange!20!black,colframe=orange!25,title=#1] } { \end{tcolorbox} }
\newenvironment{fct}[1][Fact] { \begin{tcolorbox}[colback=orange!10,coltitle=orange!20!black,colframe=orange!25,title=#1] } { \end{tcolorbox} }
\newenvironment{rmk}[1][Remark] { \begin{tcolorbox}[colback=orange!10,coltitle=orange!20!black,colframe=orange!25,title=#1] } { \end{tcolorbox} }
\newenvironment{thm}[1][Theorem] { \begin{tcolorbox}[colback=blue!10,coltitle=blue!20!black,colframe=blue!25,title=#1] } { \end{tcolorbox} }
\newenvironment{pf}[1][Proof] { \begin{tcolorbox}[colback=red!10,coltitle=red!20!black,colframe=red!25,title=#1] } { \end{tcolorbox} }

\usepackage{hyperref}
\author{Alek Westover}
\title{Cool Math Questions}

\begin{document}
\maketitle
\tableofcontents

\section{Fun Walking Questions}
\subsection{Hats}

\textbf{Binary:}
You have 100 ppl with black or white hats on. Line them up by height. Each will
in turn say their number. How many people can definietly say their number if
they decide on a strategy before hand?\\
\textbf{A more colorful question:}
Now there are 3 colors. Same question, (each person will say a color in turn from tallest to shortest, what's the most people that can survive?)


\section{Analysis}
\subsection{Function not equal to its Taylor Series, even though it's Taylor series doesn't blow up}
Let $$f(x) = \begin{cases}
	0 & x=0 \\
	e^{-1/x} & x>0
\end{cases}$$

Note that by L'Hospital's rule: $$\lim_{x\to 0} x^{-n} e^{-1/x} = 0$$
So the Taylor Series of $f$ around $0$ is $0$.
The taylor series has an infinite radius of convergence, but converges only at a single point to the function.

\subsection{Vitali Set Is not measurable}
Consider
$$\mathbb{R} / \mathbb{Q}$$
Take a point from each equivalnce class that lies in the interval $[0,1]$.
Combining these points form the set $V$.

Define $V_q$ for all $q\in \mathbb{Q} \cap [-1,1]$ as 
$$V_q \defeq \{v+q : v \in V\}.$$

Now consider $$U \defeq \bigcup_{q \in \mathbb{Q}\cap [-1,1]} V_q$$

Assume for contradiction that $U$ is measurable.

Because $[0,1] \subset U \subset [-1,2]$, we have $$ 1 \le \mu(U) \le 3.$$
However, it is also true that the $V_q$ are disjoint (imagine $x \in V_q \cap
V_p$, then $x = p+v_1 = q+v_2$ for some $v_1,v_2 \in V$, but then $v_1 - v_2
\in \mathbb{Q}$ so $v_1 = v_2$ because we chose a sinlge representative from
each equivalence class. Thus $p=q$, i.e. no $x$ can belong to multiple
$V_q$'s.)
But then, the killer blow is countable additivity of measure,
$$ \mu(U) = \sum_{q\in \mathbb{Q}} \mu(V_q).$$
$$\mu(V_q) = \mu(V) \;\;\;\forall q \in \mathbb{Q}$$
But this is really sus, because either we are adding up countably many zeros
and getting $0$ or the terms are nonzero in which case the sum in infinite.
But this contradicts our bound on $\mu(U)$.\\
Thus $U$ is not measurable.


\end{document}

