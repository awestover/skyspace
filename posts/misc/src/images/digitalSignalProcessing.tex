
\documentclass{article}[11pt]

\usepackage{amsthm}
\usepackage{amssymb}
\usepackage{amsmath}
\usepackage{mathtools}

\DeclareMathOperator{\img}{Im}
\DeclareMathOperator{\polylog}{\text{polylog}}
\DeclareMathOperator{\st}{\text{ such that }}

\newcommand{\contr}[0]{\[ \Rightarrow\!\Leftarrow \]}
\newcommand{\defeq}{\vcentcolon=}
\newcommand{\eqdef}{=\vcentcolon}

\newtheorem{fact}{Fact}
\newtheorem{definition}{Definition}
\newtheorem{remark}{Remark}
\newtheorem{proposition}{Proposition}
\newtheorem{lemma}{Lemma}
\newtheorem{corollary}{Corollary}
\newtheorem{theorem}{Theorem}

\usepackage{tcolorbox}
\newenvironment{lem}[1][Lemma] { \begin{tcolorbox}[colback=green!10,coltitle=green!20!black,colframe=green!25,title=#1] } { \end{tcolorbox} }
\newenvironment{cor}[1][Corollary] { \begin{tcolorbox}[colback=green!10,coltitle=green!20!black,colframe=green!25,title=#1] } { \end{tcolorbox} }
\newenvironment{prop}[1][Proposition] { \begin{tcolorbox}[colback=green!10,coltitle=green!20!black,colframe=green!25,title=#1] } { \end{tcolorbox} }
\newenvironment{defn}[1][Definition] { \begin{tcolorbox}[colback=orange!10,coltitle=orange!20!black,colframe=orange!25,title=#1] } { \end{tcolorbox} }
\newenvironment{fct}[1][Fact] { \begin{tcolorbox}[colback=orange!10,coltitle=orange!20!black,colframe=orange!25,title=#1] } { \end{tcolorbox} }
\newenvironment{rmk}[1][Remark] { \begin{tcolorbox}[colback=orange!10,coltitle=orange!20!black,colframe=orange!25,title=#1] } { \end{tcolorbox} }
\newenvironment{thm}[1][Theorem] { \begin{tcolorbox}[colback=blue!10,coltitle=blue!20!black,colframe=blue!25,title=#1] } { \end{tcolorbox} }
\newenvironment{pf}[1][Proof] { \begin{tcolorbox}[colback=red!10,coltitle=red!20!black,colframe=red!25,title=#1] } { \end{tcolorbox} }

\author{Alek Westover}
\title{Digital Signal Processing}

\begin{document}
\maketitle
\tableofcontents
% \begin{center}
% \begin{Large}
%   Alek Westover \\
%   \vspace{2mm}
%   Digital Signal Processing
% \end{Large}
% \end{center}

% \setcounter{section}{-1}
% \section{Outline}
% \begin{itemize}
%   \item What is a signal?
%   \item Linear algebra excursion - develop the concept of a basis for $\mathbb{R}^N$ (focussing on orthogonal bases)
%   \item What is the Fourier transform?
%   \item What is a Linear Time Invariant Opperator?
%   \item Convolution Theorem, Modulation Theorem
%   \item Filters - Ideal Filters, realizable filters, useful filters
% \end{itemize}

\section{What is a Signal?}
There are $4$ major classes of signals that I will be talking about. 
First however I shall note that I only consider discrete time signals.
This may seem like an unnaccpetable simplification because "we live in a continuous world".
This however is not so.
While some analog devices give the illusion of continuous signals, like a thermometer's reading over time, in reality we can only take measurements at discrete time intervals. 

\paragraph{Finite Length Signals}
In practice computers have finite memory, and we have finite time to perform signal processing. 
Thus, Finite Length Signals are very important to the real world.

Finite  are can be thought of in 2 different ways. 
You can think of a length $N$ finite length signal as a function from $\{0,1,\ldots, N-1\}$ to $\mathbb{C}$ or as a vector in $\mathbb{C}^N$.

Here are some useful examples of finite length signals.

The delta function:
$$
\delta[n] = 
\begin{cases}
	1 & n=0 \\
	0 & \text{else.}
\end{cases}
$$

The complex exponential:
$$x_k[n] = \exp\Big(i\frac{2\pi}{N}nk\Big)$$
This is a class of periodic signals where $k\in\{0,1,\ldots, N-1\}$. 
Note that not all sinosoids are periodic in discrete time. 
For instance the signal $$x[n] = \sin\Big(\frac{2\pi}{N}n\sqrt{2}\Big)$$ is not periodic.

\paragraph{Periodic Signals}
If a (discrete-time) signal is periodic with period $M$, then $x[n + k\cdot M] = x[n]$ for all $k$. Note that such a signal has exactly as much information as $1$ period of the signal. 
This is a very natural way of embedding a finite lenght signal in an infinite length signal.
Given a finite length signal $x[n]$ of length $N$ we can construct the periodic extension as 
$$\tilde{x}[n] = x[n\mod N].$$

% Note that the following is periodic
% $$\tilde{x}[n] \defeq \sum_{k\in \mathbb{Z}} x[n+k\cdot M].$$

\paragraph{Finite Support Signals}
Another way to embed a finite length signal in an infinite length signal is to simply pad the signal with an infinite number of $0$s.
This is denoted
$$
\bar{x}[n] = 
\begin{cases}
	x[n] & n \in \{0,1,\ldots N-1\}\\
	0 & \text{else.}
\end{cases}
$$

\paragraph{Infinite Length Signals}
Infinite Lenght signals are useful mostly as a theoretical tool, and an ideal that we can approximate with finite length signals.


\section{Linear Algebra Excursion - Developing the concept of a basis for $\mathbb{R}^N$}
Example: expressing points coordinates relative to rotated standard bases vectors in the plane.
If the basis is orthogonal it is easy to get the coordinate transformation, and Parseval's theorem applies.

\section{The Fourier Transform}
The Fourier Transform is a very special change of basis. It takes a signal in the time domain, and expresses it in the frequency domain.
This means that we are expressing the signal as an infinite sum of weighted sinosoids of differing freqeuncies.

The DFT is defined as 
$$X[k] = \sum_{n=0}^{N-1} x[n] e^{-i\frac{2\pi}{N}nk}.$$
This is invertible. The IDFT is 
$$x[n] = \frac{1}{N}\sum_{k=0}^{N-1} X[k] e^{i\frac{2\pi}{N}nk}.$$
Note that $$e^{-i\frac{2\pi}{N}nk} = e^{i\frac{2\pi}{N} (N-nk)}$$
by the $2\pi$ periodicity of the complex exponential $e^{i\omega}$.
This means that it we could have used either direction of listing coefficients for the basis.
It is standard to use the one that I chose.

The DFS is just the periodic extension of this and is the version of the fourier transform for periodic sequences.

The DTFT is defined as 
$$X(e^{j\omega}) = \sum_{n\in\mathbb{Z}} x[n] e^{-j\omega n}.$$
And the IDTFT is 
$$x[n] = \frac{1}{2\pi}\int_{-\pi}^\pi X(e^{j\omega}) e^{j\omega n} d\omega.$$

The intervability of these transforms highlights the fact that a signal can be represented in either the time or frequency domain.

There are lots of important properties of the Fourier Transform.

Here are a few:
\begin{itemize}
	\item Linearity (DTFT is linear opperator)
	\item Time shift: $$x[n-n_0] \implies e^{-j\omega n_0} X(e^{j\omega})$$
	\item Modulation $$e^{j \omega_0 n}x[n] \implies X(e^{j(w-w_0)})$$
	\item Time reversal: $$x[-n] \implies X(e^{-j\omega})$$
	\item Conjugation: $$x^*[n] \implies X^*(e^{-j\omega})$$
\end{itemize}


\section{FFT}
DFT can be computed in $O(N\log N)$ time (much better than the naive $O(N^2)$ algorithm).
This is because of the following remarkable fact:

Let $x[n]$ be a signal of length $2N$. 

Let $$x_e[n] = x[2n]\,\, n=0,1,\ldots, N-1.$$

Let $$x_o[n] = x[2n+1]\,\, n=0,1,\ldots, N-1.$$

Observe the following about the DFT of $x[n]$.

$$X[k] = \sum_{n=0}^{2N-1} x[n]e^{-i\frac{2\pi}{2N}nk}.$$
$$X[k] = \sum_{n=0}^{N-1} x_e[n] e^{-i\frac{2\pi}{2N}2nk} + x_o[n]e^{-i\frac{2\pi}{2N}(2n+1)k} .$$
$$X[k] = \sum_{n=0}^{N-1} x_e[n] e^{-i\frac{2\pi}{N}nk} + e^{-i\frac{\pi}{N}k} \sum_{n=0}^{N-1} x_o[n]e^{-i\frac{2\pi}{N}nk} .$$

This is great, because we reduced the problem of finding the DFT of a length $N$ signal to computing the DFT of $2$ length $N/2$ signals (and then adding them with a weight on the odd DFT).
There are more optimizations that go into the FFT that I haven't discussed here, but this is the gist of it.

\section{LTIs}
A linear time invariant opperator is an opperator that is linear and time invariant.
This means that they are described entirely by their impulse response, because we can shift, scale, and add deltas to get any signal, i.e. for all signals $x[n]$,
$$x[n] = \sum_{k\in\mathbb{Z}}x[k]\delta[n-k].$$

We define convolution 
$$(x[n]*h[n])[n] = \sum_{k\in\mathbb{Z}}x[k]\cdot h[n-k].$$
If $h[n]$ is the impulse response of an LTI convolving $h[n]$ with the input signal evaluates the LTI.

Intuitively we are time reversing $h$ and then repeatedly centering $h$ at different values and taking inner products.

\section{Filters}
Low pass filter is cool. But its impulse response is sinc, which is no good.

\end{document}

