\documentclass{article}[11pt]
\usepackage[subtle]{savetrees}
\usepackage[left=1in, right=1in, top=1in, bottom=1in]{geometry}

\usepackage{amsthm}
\usepackage{amssymb}
\usepackage{amsmath}
\usepackage{mathtools}

\usepackage{fancyhdr}
\pagestyle{fancy}
\lhead{Alek Westover}
\rhead{}

\usepackage{hyperref}

\DeclareMathOperator{\E}{\mathbb{E}}
\DeclareMathOperator{\Var}{\text{Var}}
\DeclareMathOperator{\img}{Im}
\DeclareMathOperator{\polylog}{\text{polylog}}
\DeclareMathOperator{\st}{\text{ such that }}
\newcommand{\norm}[1]{\left\lVert#1\right\rVert}
\newcommand{\interior}[1]{%
  {\kern0pt#1}^{\mathrm{o}}%
}

%** VECTOR NOTATION
\newcommand{\mb}{\mathbf}
\newcommand{\x}{\mathbf{x}}
\newcommand{\y}{\mathbf{y}}
\newcommand{\z}{\mathbf{z}}
\newcommand{\f}{\mathbf{f}}
\newcommand{\n}{\mathbf{n}}
\newcommand{\p}{\mathbf{p}}
\renewcommand{\k}{\mathbf{k}}
\renewcommand{\d}{\mathrm{d}} %straight d for integrals
\newcommand{\De}{\Delta}
\renewcommand{\Re}{\mathrm{Re}}
\renewcommand{\Im}{\mathrm{Im}}
\newcommand{\ran}{\mathrm{ran}}

%** SETS
\newcommand{\set}[1]{\mathbb{#1}}
\newcommand{\curly}[1]{\mathcal{#1}}
\newcommand{\goth}[1]{\mathfrak{#1}}
\newcommand{\setof}[2]{\left\{ #1\; : \;#2 \right\}}
\newcommand{\cc}{\subseteq\subseteq}
\newcommand{\R}{\set{R}}
\newcommand{\C}{\set{C}}
\newcommand{\Z}{\set{Z}}
\newcommand{\D}{\curly{D}}
\renewcommand{\S}{\set{S}}
\newcommand{\T}{\set{T}}

\newcommand{\contr}[0]{\[ \Rightarrow\!\Leftarrow \]}
\newcommand{\defeq}{\vcentcolon=}
\newcommand{\eqdef}{=\vcentcolon}

\newtheorem{fact}{Fact}
\newtheorem{definition}{Definition}
\newtheorem{remark}{Remark}
\newtheorem{proposition}{Proposition}
\newtheorem{lemma}{Lemma}
\newtheorem{corollary}{Corollary}
\newtheorem{theorem}{Theorem}

\usepackage{xcolor}
\usepackage{amsthm}
\usepackage{framed}
\theoremstyle{plain}% default


\newtheorem{prototheorem}{Theorem}[section]
\newenvironment{thm}
   {\colorlet{shadecolor}{blue!20}\begin{shaded}\begin{prototheorem}}
   {\end{prototheorem}\end{shaded}}

\newtheorem{protodefinition}[prototheorem]{Definition}
\newenvironment{defn}
   {\colorlet{shadecolor}{black!=5}\begin{shaded}\begin{protodefinition}}
   {\end{protodefinition}\end{shaded}}

\newtheorem{protolemma}[prototheorem]{Lemma}
\newenvironment{lem}
   {\colorlet{shadecolor}{green!15}\begin{shaded}\begin{protolemma}}
   {\end{protolemma}\end{shaded}}

\newtheorem{protocorollary}[prototheorem]{Corollary}
\newenvironment{cor}
   {\colorlet{shadecolor}{green!15}\begin{shaded}\begin{protocorollary}}
   {\end{protocorollary}\end{shaded}}

\newtheorem{protoproposition}[prototheorem]{Proposition}
\newenvironment{prop}
   {\colorlet{shadecolor}{green!15}\begin{shaded}\begin{protoproposition}}
   {\end{protoproposition}\end{shaded}}

 \newtheorem{protoexample}[prototheorem]{Example}
\newenvironment{ex}
   {\colorlet{shadecolor}{red!15}\begin{shaded}\begin{protoexample}}
   {\end{protoexample}\end{shaded}}

\newtheorem{protoremark}[prototheorem]{Remark}
\newenvironment{rmk}
   {\colorlet{shadecolor}{red!15}\begin{shaded}\begin{protoremark}}
   {\end{protoremark}\end{shaded}}


\author{Alek Westover}
\title{Notes}
\date{someYear}

\begin{document}
\maketitle
Functional analysis is a really cool branch of mathematics. Imagine analysis, and then add linear algebra. And then make it all {\color{red}\textbf{INFINTITE DIMENSIONAL}}. It's super geometrical and stuff.

\tableofcontents

\section{the first part}
\subsection{a cool proof}
\begin{rmk}
  asdlfasdf
\end{rmk}

\begin{thm}
  (The projection theorem). Let $H$ be a Hilbert space, let $M\subset H$ be a closed subspace of $H$. Let $x \in H$. Then,
  $$\exists! m_0 \in M \text{ such that } \norm{x-m_0} = \inf_{m\in M} \norm{x-m}$$
  and $m_0$ is characterized by 
  $$x-m_0 \in M^\perp \text{ i.e. } x-m_0 \perp m \quad \forall m \in M$$
\end{thm}

\begin{proof}
Lorem ipsum dolor sit amet, consetetur sadipscing elitr, sed diam nonumy eirmod
Lorem ipsum dolor sit amet, consetetur sadipscing elitr, sed diam nonumy eirmod
Lorem ipsum dolor sit amet, consetetur sadipscing elitr, sed diam nonumy eirmod
Lorem ipsum dolor sit amet, consetetur sadipscing elitr, sed diam nonumy eirmod
Lorem ipsum dolor sit amet, consetetur sadipscing elitr, sed diam nonumy eirmod
Lorem ipsum dolor sit amet, consetetur sadipscing elitr, sed diam nonumy eirmod
Lorem ipsum dolor sit amet, consetetur sadipscing elitr, sed diam nonumy eirmod
\end{proof}


\begin{ex}
  (The projection theorem). Let H be a asdfklasdflkasd
  $$\norm{x-m}$$
  bro bro
  $$\text{guass}$$
  
\end{ex}
\begin{lem}
  (The projection theorem). Let H be a asdfklasdflkasd
  $$\norm{x-m}$$
  bro bro
  $$\text{guass}$$
  
\end{lem}

\begin{cor}
  (The projection theorem). Let H be a asdfklasdflkasd
  $$\norm{x-m}$$
  bro bro
  $$\text{guass}$$
  
\end{cor}

\begin{defn}
 asdfasdf 
\end{defn}
\end{document}

